%%% mode: latex
%%% TeX-master: "wireless-security"

\documentclass[journal, compsoc]{IEEEtran}

\usepackage{silence}
\WarningFilter{breakurl}{You are using breakurl while processing via pdflatex}
\WarningFilter{hyperref}{Token not allowed in a PDF string (PDFDocEncoding)}

\usepackage[utf8]{inputenc}
\usepackage[T1]{fontenc}
\usepackage[english]{babel}
\usepackage[autostyle]{csquotes}
\usepackage[cmex10]{amsmath}
\usepackage{xmpincl}
\usepackage{filecontents}
\usepackage{siunitx}
\usepackage[hyphens]{url}
\usepackage[hyphenbreaks]{breakurl}
\usepackage{doi}
\usepackage[backend=biber,style=ieee,citestyle=ieee,sortcites,url=false]{biblatex}

\interdisplaylinepenalty=2500
\setcounter{biburlnumpenalty}{3000}
\setcounter{biburllcpenalty}{6000}
\setcounter{biburlucpenalty}{9000}

%%%%%%%%%%%%%%%%%%%%%%%%%%%%%%%%%%%%%%
%             License                %
%%%%%%%%%%%%%%%%%%%%%%%%%%%%%%%%%%%%%%

\begingroup\newif\ifmy{}
\IfFileExists{copyright.xmp}{}{\mytrue}
\ifmy{}
\begin{filecontents}{copyright.xmp}
<?xpacket begin='' id=''?>
<x:xmpmeta xmlns:x='adobe:ns:meta/'>
  <rdf:RDF xmlns:rdf='http://www.w3.org/1999/02/22-rdf-syntax-ns#'>
    <rdf:Description rdf:about=''
                     xmlns:xapRights='http://ns.adobe.com/xap/1.0/rights/'>
      <xapRights:Marked>True</xapRights:Marked>
    </rdf:Description>
    <rdf:Description rdf:about=''
                     xmlns:xapRights='http://ns.adobe.com/xap/1.0/rights/'
                     >
      <xapRights:UsageTerms>
	<rdf:Alt>
          <rdf:li xml:lang='x-default' >This work is licensed under a &lt;a rel=&#34;license&#34; href=&#34;http://creativecommons.org/licenses/by-sa/4.0/&#34;&gt;Creative Commons Attribution-ShareAlike 4.0 International License&lt;/a&gt;.</rdf:li>
          <rdf:li xml:lang='en_US' >This work is licensed under a &lt;a rel=&#34;license&#34; href=&#34;http://creativecommons.org/licenses/by-sa/4.0/&#34;&gt;Creative Commons Attribution-ShareAlike 4.0 International License&lt;/a&gt;.</rdf:li>
          <rdf:li xml:lang='en' >This work is licensed under a &lt;a rel=&#34;license&#34; href=&#34;http://creativecommons.org/licenses/by-sa/4.0/&#34;&gt;Creative Commons Attribution-ShareAlike 4.0 International License&lt;/a&gt;.</rdf:li>
          </rdf:Alt>
      </xapRights:UsageTerms>
    </rdf:Description>
    <rdf:Description rdf:about=''
                     xmlns:dc='http://purl.org/dc/elements/1.1/'>
      <dc:title>
	<rdf:Alt>
          <rdf:li xml:lang='x-default'>Multi-Surface Attack on Wireless Security</rdf:li>
          <rdf:li xml:lang='en_US'>Multi-Surface Attack on Wireless Security</rdf:li>
	</rdf:Alt>
      </dc:title>
    </rdf:Description>
    <rdf:Description rdf:about=''
      xmlns:cc='http://creativecommons.org/ns#'>
      <cc:license rdf:resource='http://creativecommons.org/licenses/by-sa/4.0/'/>
    </rdf:Description>
    <rdf:Description rdf:about=''
      xmlns:cc='http://creativecommons.org/ns#'>
      <cc:attributionName>Brian Ridings, Tyler Romeo, and Neal Trischitta</cc:attributionName>
    </rdf:Description>

  </rdf:RDF>
</x:xmpmeta>
<?xpacket end='r'?>
\end{filecontents}
\fi\endgroup
\includexmp{copyright}

%%%%%%%%%%%%%%%%%%%%%%%%%%%%%%%%%%%%%%
%           Bibliography             %
%%%%%%%%%%%%%%%%%%%%%%%%%%%%%%%%%%%%%%

\begingroup\newif\ifmy{}
\IfFileExists{references.bib}{}{\mytrue}
\ifmy{}
\begin{filecontents}{references.bib}
@article{1389197,
  journal = {ANSI/IEEE Std 802.11, 1999 Edition (R2003)},
  title = {IEEE Standard for Information Technology- Telecommunications and Information Exchange Between Systems-Local and Metropolitan Area Networks- Specific Requirements- Part 11: Wireless LAN Medium Access Control (MAC) and Physical Layer (PHY) Specifications},
  year = {2003},
  month = {},
  pages = {i-513},
  doi = {10.1109/IEEESTD.2003.95617},
}

@article{1438730,
  journal = {IEEE Std 802.1X-2004 (Revision of IEEE Std 802.1X-2001)},
  title = {IEEE Standard for Local and Metropolitan Area Networks Port-Based Network Access Control},
  year = {2004},
  pages = {0\_1-169},
  doi = {10.1109/IEEESTD.2004.96095},
}

@article{4100091,
  journal = {ISO/IEC 8802-11, Second edition: 2005/Amendment 6 2006: IEEE STD 802.11i-2004 (Amendment to IEEE Std 802.11-1999)},
  title = {ISO/IEC International Standard - Information Technology Telecommunications and Information Exchange Between Systems Local and Metropolitan Area Networks Specific Requirements Part 11: Wireless LAN Medium Access Control (MAC) and Physical Layer (PHY) Specifications Amendment 6: Medium Access Control (MAC) Security Enhancements},
  year = {2004},
  month = {July},
  pages = {c1-178},
  doi = {10.1109/IEEESTD.2004.311922},
}

@inproceedings{Fluhrer:2001:WKS:646557.694759,
 author = {Fluhrer, Scott R. and Mantin, Itsik and Shamir, Adi},
 title = {Weaknesses in the Key Scheduling Algorithm of RC4},
 booktitle = {Revised Papers from the 8th Annual International Workshop on Selected Areas in Cryptography},
 series = {SAC '01},
 year = {2001},
 isbn = {3-540-43066-0},
 pages = {1--24},
 numpages = {24},
 url = {http://dl.acm.org/citation.cfm?id=646557.694759},
 acmid = {694759},
 publisher = {Springer-Verlag},
 address = {London, UK, UK},
}

@inproceedings{Tews:2007:BBW:1784964.1784983,
 author = {Tews, Erik and Weinmann, Ralf-Philipp and Pyshkin, Andrei},
 title = {Breaking 104 Bit WEP in Less Than 60 Seconds},
 booktitle = {Proceedings of the 8th International Conference on Information Security Applications},
 series = {WISA'07},
 year = {2007},
 isbn = {3-540-77534-X, 978-3-540-77534-8},
 location = {Jeju Island, Korea},
 pages = {188--202},
 numpages = {15},
 url = {http://dl.acm.org/citation.cfm?id=1784964.1784983},
 acmid = {1784983},
 publisher = {Springer-Verlag},
 address = {Berlin, Heidelberg},
}

@inproceedings{Wu:2006:STA:1124772.1124863,
 author = {Wu, Min and Miller, Robert C. and Garfinkel, Simson L.},
 title = {Do Security Toolbars Actually Prevent Phishing Attacks?},
 booktitle = {Proceedings of the SIGCHI Conference on Human Factors in Computing Systems},
 series = {CHI '06},
 year = {2006},
 isbn = {1-59593-372-7},
 location = {Montr\&\#233;al, Qu\&\#233;bec, Canada},
 pages = {601--610},
 numpages = {10},
 url = {http://doi.acm.org/10.1145/1124772.1124863},
 doi = {10.1145/1124772.1124863},
 acmid = {1124863},
 publisher = {ACM},
 address = {New York, NY, USA},
 keywords = {e-commerce, user interface design, user study, world wide web and hypermedia},
}

@incollection{Halvorsen:2009,
  year = {2009},
  isbn = {978-3-642-04765-7},
  booktitle = {Identity and Privacy in the Internet Age},
  volume = {5838},
  series = {Lecture Notes in Computer Science},
  editor = {Jøsang, Audun and Maseng, Torleiv and Knapskog, SveinJohan},
  doi = {10.1007/978-3-642-04766-4_9},
  title = {An Improved Attack on TKIP},
  url = {http://dx.doi.org/10.1007/978-3-642-04766-4_9},
  publisher = {Springer Berlin Heidelberg},
  author = {Halvorsen, FinnM. and Haugen, Olav and Eian, Martin and Mjølsnes, StigF.},
  pages = {120-132},
  language = {English},
}

@article{viehbock2011brute,
  title={Brute forcing wi-fi protected setup},
  author={Viehb{\"o}ck, Stefan},
  journal={Wi-Fi Protected Setup},
  year={2011}
}

@techreport{aboba2004extensible,
  title={Extensible authentication protocol (EAP)},
  author={Aboba, Bernard and Blunk, Larry and Vollbrecht, John and Carlson, James and Levkowetz, Henrik and others},
  year={2004},
  institution={RFC 3748, June}
}

@techreport{harkins2010extensible,
  title={Extensible Authentication Protocol (EAP) Authentication Using Only a Password},
  author={Harkins, D and Zorn, G},
  year={2010},
  institution={RFC 5931, August}
}

@techreport{zorn1998microsoft,
  title={Microsoft PPP CHAP Extensions},
  author={Zorn, G and Cobb, S},
  year={1998},
  institution={RFC 2433, October}
}

@article{zorn2000microsoft,
  title={Microsoft PPP CHAP extensions, version 2},
  author={Zorn, Glen},
  year={2000},
  institution={RFC 2759, January}
}

@online{marlinspike2012divide,
  title={Divide and Conquer: Cracking MS-CHAPv2 with a 100\% success rate},
  author={Moxie Marlinspike},
  date={2012-07-29},
  url={https://www.cloudcracker.com/blog/2012/07/29/cracking-ms-chap-v2/},
  urldate={2014-12-10},
}

@online{tpwr703n,
  title={TP-Link TL-WR703N},
  author={mandrawes},
  organization={OpenWrt},
  date={2014-11-29},
  url={http://wiki.openwrt.org/toh/tp-link/tl-wr703n?rev=1417238011},
  urldate={2014-12-10},
}

@online{keeble2014passive,
  title={Passive WiFi Tracking},
  author={Edward Keeble},
  date={2014-02-26},
  url={http://edwardkeeble.com/2014/02/passive-wifi-tracking/},
  urldate={2014-12-10},
}

@online{hak5pineapple,
  title={Wifi Pineapple / Jasager},
  author={Robin Wood},
  date={2014-12-10},
  url={https://forums.hak5.org/index.php?/forum/64-wifi-pineapple-jasager/},
  urldate={2014-12-10},
}

@online{dunstan2010attacking,
  title={Attacking and Securing PEAP},
  author={Patrick Dunstan},
  date={2010-05-17},
  url={http://www.defenceindepth.net/2010/05/attacking-and-securing-peap.html},
  urldate={2014-12-10},
}
\end{filecontents}
\fi\endgroup
\addbibresource{references.bib}

%%%%%%%%%%%%%%%%%%%%%%%%%%%%%%%%%%%%%%
%               Header               %
%%%%%%%%%%%%%%%%%%%%%%%%%%%%%%%%%%%%%%

\title{Multi-Surface Attack on Wi-Fi Security}

\author{%
  Brian~Ridings,
  Tyler~Romeo,
  and~Neal~Trischitta
  \IEEEcompsocitemizethanks{\IEEEcompsocthanksitem{} B. Ridings,
    T. Romeo, and N. Trischitta are with the Stevens Institute of
    Technology.}
  \thanks{Copyright \texorpdfstring{\textcopyright}{(c)} 2014 Brian
    Ridings, Tyler Romeo, and Neal Trischitta. Some rights
    reserved. This work is licensed under the Creative Commons
    Attribution-ShareAlike 4.0 International License. To view a copy
    of this license, visit
    \url{http://creativecommons.org/licenses/by-sa/4.0/}.}
}
\markboth{CS-577 Cybersecurity Lab, Fall 2014}{Ridings
  \MakeLowercase{\textit{et al.}}: Multi-Surface Attack of Wi-Fi
  Security}

\begin{document}

\IEEEtitleabstractindextext{%
  \begin{abstract}
    Wireless security is critical in both personal and corporate
    environments. Almost all wireless security protocols have some
    known exploit, and only select few configurations can provide real
    security. A proof-of-concept of this fact was developed by making
    a tiny form-factor, minimal-resource using router that can
    automatically exploit multiple types of insecure wireless network
    configurations. The deliverable was a success, and demonstrated
    that using less than \$15 hardware, an attacker can exploit WEP-, WPA-TKIP-,
    EAP-MD5-, and EAP-MSCHAPv2-protected networks. Additionally,
    networks with unsigned or self-signed X.509 certificates could be
    trivially spoofed, allowing replay of client credentials. Router
    and operating system developers are deemed responsible for
    improving their products by disabling insecure protocols and
    encouraging user best practices.
  \end{abstract}
  \begin{IEEEkeywords}
    Wireless systems, network-level security and protection, unauthorized access
  \end{IEEEkeywords}
}

\maketitle

%%%%%%%%%%%%%%%%%%%%%%%%%%%%%%%%%%%%%%
%               Paper                %
%%%%%%%%%%%%%%%%%%%%%%%%%%%%%%%%%%%%%%

\section{Introduction}

\IEEEPARstart{W}{ireless} security is critical infrastructure in
high-confidentiality or high-integrity networks. Eavesdropping and
packet alteration attacks are significantly easier on a wireless
medium, and thus traditional security mechanisms may not be enough due
to the inherent insecurity of the channel. The motivation for our
project was to demonstrate vulnerabilities in Wi-Fi security. For this
project, researchers extended OpenWRT, an open Linux-based firmware
that allows taking a high level perspective while not having to be
concerned about the drivers and other inconsequential parts of the
project. It also gave researchers the ability for full
customization of the device and its components. The deployment of the
Wi-Fi attack framework will attempt to intercept client credentials
from the wireless interface by actively listening for probe requests
from clients. In addition, our framework will be able to compute basic
cracking of wireless protocols with known vulnerabilities.

The purpose of the experiment is to demonstrate as a proof-of-concept
exactly how many vulnerabilities exist in real-world wireless networks
and how easy it is for an attacker to infiltrate an insecure network
using minimal resources and opportunity. The first step to finding a
solution is to clearly identify the problem, and it is important that
users understand the liability they are incurring when using insecure
wireless security protocols.

Some related Work is the Hak5 Wifi Pineapple~\cite{hak5pineapple}. The
Pineapple responds to client probe requests and lets it connect with
that SSID on an open access point. The goal of this new device to not
respond to probe request, but instead to create an access point and
then wait for the client to initiate contact. Another goal is to allow for
mutable and different encryption types, whereas the Pineapple only
makes open access points. The Pineapple wants the user to connect so
that a man-in-the-middle attack can be performed, but the new device
wants to first capture credentials and other information that may be
sensitive to the client, without the client initiating the connection.

This paper is organized as follows: \autoref{sec:overview} gives an
overview of various wireless security protocols in use in modern
networks, and the vulnerabilities and exploits associated with those
protocols; \autoref{sec:design} outlines the design and implementation
of the proof-of-concept hardware deliverable demonstrating the
aforementioned exploits; and finally \autoref{sec:results} explains
the results of testing the final product and what was learned from
those results.

\section{Wireless Security Overview}
\label{sec:overview}

Securing wireless networks is a dubious task, to the point where some
companies simply do not allow wireless access to corporate services at
all. Wireless networks are generally secured using one of two security
standards: either IEEE 802.11--1999 (WEP) or IEEE 801.11i-2004 (WPA and
WPA2), the latter optionally combined with Wi-Fi Protected Security
(WPS) or IEEE 802.1X-2004.

\subsection{WEP}
\label{sec:overview-wep}

Wired Equivalent Privacy, often acronymed as WEP, was the first
security algorithm developed for Wi-Fi, and was initially specified as
part of the original IEEE 802.11 standard as released in 2003
\cite{1438730}. As the name implies, the protocol was designed
primarily to provide privacy and confidentiality for wireless
communications, i.e., ensure that other clients could not listen in on
traffic being sent to the access point. WEP even has an `Open' mode,
where traffic is encrypted but clients do not have to authenticate to
the access point. In fact, the only method of authentication or access
control provided by WEP is pre-shared key authentication, which relies
on using the key in an RC4 cipher when encrypting challenge text from
the access point.

WEP is thoroughly broken. As an emphasis of this fact, using `Open'
mode, where no authentication is performed, is actually more secure
than using a pre-shared key. (The reason behind this has to do with
deriving the keystream by capturing the initial challenge.) Among the
many issues with the protocol are use of RC4, a stream cipher, which
requires using a new key every time, and a weak 23-bit initialization
vectors that are open to repetition. These vulnerabilities resuled in
the Fluher, Mantin, and Shamir attack
\cite{Fluhrer:2001:WKS:646557.694759}, which, including improvements
over the years, allows a 95\% chance recovery of a 104-bit WEP key
using only 85,000 captured network packets
\cite{Tews:2007:BBW:1784964.1784983}. This attack can be achieved on
consumer hardware in just a few minutes.

\subsection{WPA}
\label{sec:overview-wpa}

WEP has since been deprecated and superseded by Wi-Fi Protected
Access. The first version of WPA was released by the Wi-Fi Alliance in
the IEEE 802.11i draft standard as a stop-gap measure for replacing
WEP, at least until the IEEE 802.11 working group could flesh out the
more final version of the protocol. That final version was released as
IEEE 802.11i-2004 and is commonly referred to was WPA2
\cite{4100091}. The first version of WPA used the Temporal Key
Integrity Protocol (TKIP). Although TKIP uses a 128-bit key and
discourages attacks with a new key mixing function and message
integrity checks, it uses some of the same techniques as WEP
encryption, e.g., using CRC32 for checksums, and thus is vulnerable in
many of the same ways. Halvorsen, et.~al~\cite{Halvorsen:2009}
demonstrated being able to recover upto 596 bytes of the TKIP
keystream in under twenty minutes. Eventually this was fixed in WPA2
with the advent of the Counter Mode CBC-MAC Protocol (CCMP or
AES-CCMP), which is considered secure as of the writing of this
paper. CCMP works by using CBC-MAC to encrypt the data and compute the
integrity tag, and then encrypting that tag in CTR mode, thus
achieving authenticated encryption. Any attack on CCMP would imply an
attack on AES itself.

WPA also, unlike WEP, has more support dedicated to access control. In
addition to using a pre-shared key, two other methods of client
authentication may be used: Wi-Fi Protected Setup (WPS) and IEEE
802.1X.

\subsection{WPS}
\label{sec:overview-wps}

WPS was created by the Wi-Fi Alliance in 2006. It is not an IEEE
standard, but nonetheless it is supported on many consumer routers. It
was created to allow non-technology-savvy users to authenticate their
wireless devices more easily. The protocol works over EAP, and
involves having the user either push a button on the router or enter a
preset PIN on the device. The enrollee (client) and the registrar
(access point) negotiate configuration data and then reconnect using
the new configuration. The WPS protocol has since been broken due to a
timing attack on the PIN authentication, which allows a brute force
attack to be completed in under four hours
\cite{viehbock2011brute}. To make things worse, some routers are
misleading, and even if a user disables WPS in the user interface, it
remains enabled and the router will still negotiate with
clients. Fortunately, some router manufacturers have released firmware
updates that fix this issue, while others have implemented
rate-limiting on the WPS mechanism to stop brute-force attacks.

\subsection{802.1X}
\label{sec:overview-8021x}

Contrary to WPS, 802.1X is not an authentication protocol, but an
wrapping of an existing authentication framework. IEEE~802.1X-2007 is
a method of encapsulating the Extensible Authentication Protocol (or
EAP, as defined in RFC~3748~\cite{aboba2004extensible}) in IEEE~802
packets~\cite{1438730}. The protocol is sometimes referred to as
EAPOL, or EAP over LAN, and was originally designed for IEEE~802.3
Ethernet, but was extended to wireless in 2004. The 802.1X protocol
itself is not of much interest for security research, but the
underlying EAP is. As aforementioned, EAP is an authentication
framework that can be used with many different authentication methods,
some more secure than others.

Common authentication methods available on EAP are: EAP-MD5, EAP-PWD,
EAP-MSCHAP, EAP-TLS, EAP-TTLS, and PEAP\@. There are many others, but
only these are both widely supported natively by clients and are
commonly used by network administrators.

Starting at the bottom, EAP-MD5 involves a pre-shared key that, as the
name implies, is combined in MD5 to authenticate the
client~\cite{aboba2004extensible}. Other than the various
vulnerabilities inherited from using MD5, this protocol does not have
any server authentication, and thus is blatantly vulnerable to
man-in-the-middle attacks using a fake access point. EAP-MD5 was
officially deprecated in Windows Vista. EAP-PWD improves upon EAP-MD5
by using a secure hash function, specifically HMAC-SHA-256, and
authenticating both the server and the
client~\cite{harkins2010extensible}. While the key exchange protocol
is resistant to attack, this protocol may still suffer from users
setting low-entropy passwords or passwords that can be guessed using a
dictionary attack.

A more commonly used protocol is EAP-MSCHAP (either version 1 as
defined in RFC 2433~\cite{zorn1998microsoft} or version 2 as defined
in RFC 2759~\cite{zorn2000microsoft}). It involves using Microsoft's
customized version of the Challenge-Handshake Authentication Protocol,
and uses a three-way handshake using SHA1, MD4, and DES (together) to
authenticate both the client and the server. As the reader may be able
to infer, the protocol is needlessly complicated and, as demonstrated
by Marlinspike in 2012~\cite{marlinspike2012divide}, can be broken by
cracking a single DES key and then cracking the resulting MD4
hash. Since both tasks are trivial on modern hardware, MS-CHAP is
considered broken.

Finally, there are EAP-TLS, EAP-TTLS, and PEAP, which, with respect to
the protocol, are considered secure. The first uses TLS to
authenticate both the server and client. X.509 certificates are issued
to clients and servers, all signed by a common certificate authority
(CA). The latter two are very similar, and do not require the client
to have its own X.509 certificate. Instead, they establish the TLS
connection as a tunnel, and then lets the client authenticate using
further protocols that are tunneled inside the TLS stream.

PEAP specifically, as developed my Microsoft, uses EAP once again
inside the encrypted TLS tunnel. In other words, it is a second
iteration of EAP inside a tunnel established via the first iteration
of EAP.\@ The catch is that each version of PEAP (there are two) only
supports a single authentication mechanism for the inner iteration of
EAP.\@ PEAPv0 uses EAP-MSCHAPv2, and PEAPv1 uses EAP-GTC, an alternative
to MSCHAPv2 developed by Cisco. With PEAP, since the inner
authentication protocol is run inside a TLS tunnel, it remains secure
even despite the vulnerabilities of
MS-CHAP~\cite{dunstan2010attacking}.\footnote{If the attacker can
  perform a man-in-the-middle attack between the access point and the
  RADIUS server used to authenticate, then a new attack on PEAP
  becomes possible, but this paper does not consider that attack since
  the primary threat model involves an outside attacker attempting to
  gain access to the network.}

However, while all of these protocols are indeed the most secure out
of all the EAP options, they still inherit vulnerabilities from the
X.509 trust model. The server's certificate must be signed by a
trusted CA, and users must be instructed not to trust unsafe
certificates.  Unfortunately, this can be near impossible since:
certificates for use network-wide in 802.1X systems are
extraordinarily expensive and are not affordable to smaller entities;
and even modern operating systems, like Windows 8 and OS X, provide
trivial and almost entirely ignorable warnings when a network provides
an untrusted server certificate.  The only method of protection is to
go multiple user interface levels deep into advanced network settings
and toggle a flag disallowing unsafe certificates.  As of the writing
of this paper, the wireless network at the Stevens Institute of
Technology does not use a trusted certificate, and the Information
Technology department explicitly instructs users to ignore warnings
and connect anyway.

\section{Design and Execution}
\label{sec:design}

\subsection{Threat Model}
\label{sec:design-model}

Overall, there are numerous vulernabilities in all of the available
security protocols for Wi-Fi. The ones more readily exploitable, as
listed above, are:

\begin{itemize}
\item Sniff traffic for WEP-secured network and brute-force the
  keystream.
\item Perform a man-in-the-middle attack on a WPA-TKIP-secured
  network.
\item Brute-force the PIN for a WPS-secured network.
\item Use dictionaries or brute-force to guess a weak WPA-PSK or
  EAP-PWD password.
\item Impersonate an Access Point of an EAP-MD5-secured network and
  crack the client credentials.
\item Crack the DES key and the resulting MD4 hash of a key exchange
  on an EAP-MSCHAP-secured network.
\item Provide a fake server TLS certificate on an EAP-TLS- or
  EAP-TTLS-secured network and social engineer the user into accepting
  it.
\item Provide a fake server TLS certificate on a PEAP-secured network
  and then crack the inner MS-CHAPv2 exchange as described before.
\end{itemize}

Many of these attacks allow complete breaks of security in the entire
wireless network. Looking at the system from the perspective of the
attacker, there are a number of assets that might be of
interest. Specifically they are:

\begin{description}
\item[Network access] \hfill \\
  The network may contain resources that only authenticated clients
  can be allowed to access. If the wireless network provides access to
  the entire LAN, including business-critical resources, this can pose
  a high-level risk to the confidentiality and integrity of those
  resources, and a medium-level risk to availability depending on the
  services running. On the other end of the spectrum, even if the
  wireless network is only provided for personal use, it could still
  pose a low-level risk for integrity, specifically impersonation.

\item[Client credentials] \hfill \\
  Some wireless attacks allow cracking of the client's original
  credentials. This gives the attacker persistent access to the
  network under the client's identity, thus excaberating any risks
  aforementioned concerning network access.

\item[Network traffic] \hfill \\
  In limited cases where the attacker gains the client credentials,
  and in cases where the attacker performs a man-in-the-middle attack
  on the client, the attacker can sniff and later data sent to and
  from the actual network. This poses a high-level risk to
  confidentiality and integrity if any mission-critical information is
  communicated over that network.
\end{description}

As a further note, some assumptions are made about the system when
making these attacks. Specifically, the attacker must be in an
opportune location to launch these attacks. A WEP key cannot be
cracked if you are not in range of the network in order to capture
traffic. As a result, it is a requirement that this device be portable
and unnoticeable, thus facilitating easy, persistent planting of the
device in a suitable ground zero. In addition, if the network is
protected by EAP-TLS, EAP-TTLS, or PEAP, the assumption needs to be
made that either the server certificate is not signed by a trusted
authority or that users are either not knowledgeable enough or not
caring enough to properly validate the certificate every time they
connect to the network. In most situations, these two assumptions can
be made trivially.

\subsection{Hardware Design}
\label{sec:design-hardware}

In order to facilitate easy placement, a tiny form factor router was
chosen for this experiment: the TP-Link TL-WR703N~\cite{tpwr703n}. The
device measures \SI{5.7}{\cm} by \SI{5.7}{\cm} by \SI{1.8}{\cm}
(slighty larger with a low-profile USB storage device plugged in), and
has an Atheros AR7240 CPU at \SI{400}{\MHz} and an Atheros AR9331
chipset, with integrated 802.11b/g/n wireless.

One of the issues with the device was power. Power is supplied via
MicroUSB, but a power source is needed for the router to remain
operational. For an attacker, plugging into an outlet may not be an
option if the router is to be hidden, and thus a portable batter must
be connected to the router when it is planted. With Wi-Fi at
\SI{18}{dB.mW}, average current is \SI{100}{\mA} at approximately
\SI{5}{\volt}, giving a power consumption of \SI{0.5}{\W}. Under
normal conditions, a small \SI{3200}{\mA.\hour} phone battery could
power the router for \SI{32}{\hour}, which is more than enough to
exploit most network vulnerabilities or capture at least one set of
client credentials. With an even more powerful battery (on the current
market going as high as \SI{25,000}{\A.\hour}), the device becomes
larger but can last more than ten days.

OpenWrt 12 `Attitude Adjustment' was used as the base system, with
various additional packages installed, such as hostapd for launching
multiple wireless interfaces simultaneously, scapy for scanning
wireless networks~\cite{keeble2014passive}, and freeradius2 for
simulating access to a RADIUS server in 802.1X-secured networks.

\subsection{Wireless Exploits}
\label{sec:design-exploits}

Some of the main purposes of the framework are to provide wireless
reconnaissance, to locate a wireless network, and to collect
information about its configuration and associated clients. The `scapy'
python library, an interactive packet manipulation program was used
for the reconnaissance framework, since it provides the ability to
forge or decode packets of a wide number of protocols. In order to use
scapy, the wireless network interface was set to monitor mode to
capture and listen to all traffic transmission from the wireless
access point. Unlike promiscuous mode, which is also used for packet
sniffing, monitor mode allows packets to be captured without having to
associate with an access point or ad hoc network first. Using `scapy',
probe requests and responses frames were captured to determine clients
attempting to actively seek any, or a particular, access point.

For discovered 802.11 PEAP/WPA networks, the device then employs a
RADIUS impersonation attack. The attack, as described earlier,
exploits the lack of certificate validation by setting up a fake
access point with a forged certificate. In addition, the device runs a
patched RADIUS server that allows all username and password
combinations to be authenticated, while simultaneously logging the
credential exchange. Combined, the client will attempt to connect and
subsequently expose the client's credentials in the resulting MS-CHAP
exchange.

\section{Results and Testing}
\label{sec:results}

The device was programmed to receive client probe requests and set up
up to six access points on demand. Access points are configured to
match the expected encryption settings of the actual access
point. Afterward the device pools for clients attaching to the access
point and removes duplicate access points that have the same SSID, but
are operating on that interface. Furthermore, old interfaces are
pruned based on the time they were created due to the upper limit of
the number of interfaces supported by the wireless driver.

There are some limitations of this setup. One is that it is hard to
delete interfaces, add interfaces, and modify them without restarting
all of the interfaces. This causes issues because any existing client
traffic is interrupted upon adding, modifying, or deleting an
interface. Currently the research team is not aware of any method of
hot-swapping wireless interfaces with the currently available drivers.

The research team also took some of the results from Spoofing
Enterprise Routers, specifically FreeRadius-wpe patch files that allow
for logging RADIUS requests and responses in order to decrypt
passwords and log them is passed in plaintext. The patches were not
made for OpenWRT, thus patches had to be applied to the version of
FreeRadius that works with OpenWRT.\@ To my knowledge this is the first
instance of a self-contained FreeRadius-wpe router in OpenWRT.\@

Otherwise, the device was able to successfully capture 802.1X client
credentials sent over PEAP-MSCHAPv2, which is the most popular
enterprise authentication protocol.

\section{Conclusion}
\label{sec:conclusion}

As demonstrated by the device, wireless security is in a broken
state. Most networks, both personal and enterprise, are open to
exploitation using various methods depending on the authentication
protocol in use.

The moral of the story is that there are only two security protocols
for Wi-Fi that are effective at mitigating threats: using WPA-CCMP-PSK
with a strong, non-guessable key; and using EAP-TLS, EAP-TTLS, or PEAP
(over 802.1X) with a trusted certificate authority and clients that
are configured to always reject untrusted certificates. Literally any
other wireless environment can be broken by consumer-grade hardware
and is entirely insecure.

The only solution to this problem is for router manufacturers to find
new ways to encourage users to utilize secure wireless technology. For
example, completely removing WEP, WPA-TKIP, and WPS from routers. By
only allowing WPA2-CCMP, using either a PSK or the secure 802.1X, the
protocol is secure and any vulnerabilities could only be introduced by
user bad practices. For WPA2 with a PSK or with PEAPv0-MSCHAPv2, this paper
does not address methods of encouraging users to create secure
passwords and follow proper password management policies, as that is a
much larger scope problem that needs to be addressed by dedicated
research. For EAP-TLS, however, operating systems developers should
make rejection of unsafe certificates the default settings and create
more prominent warnings when a user is prompted to inspect and accept
an unsafe certificate. A similar example of this problem and solution
is phishing warning toolbars in browsers, which were proven largely
ineffective and replaced with more serious and prominent, full-page
warnings~\cite{Wu:2006:STA:1124772.1124863}.

The onus is now on router and operating systems developers to disable
insecure protocols in their firmware and software, thus enforcing user
best practices and improving the general state of wireless security.

\printbibliography{}
\end{document}
